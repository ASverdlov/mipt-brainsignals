\documentclass{article}
\usepackage[utf8]{inputenc}
\usepackage[T1]{fontenc}
\usepackage[english, russian]{babel}
\usepackage{textcomp}
\usepackage{amssymb}
\usepackage{amsmath}
\usepackage{amsthm}
\usepackage{listings}
\usepackage{multicol}
\usepackage{float}
\usepackage{url}

\setlength{\topmargin}{-1.2in}
\setlength{\textheight}{10.3in}
\setlength{\oddsidemargin}{-0.4in}
\setlength{\evensidemargin}{-0.4in}
\setlength{\textwidth}{7in}
\setlength{\parindent}{0ex}

\newtheorem{thm}{Теорема}
\newtheorem{sttm}{Утверждение}

\begin{document}
\section{Аннотация}
	\iffalse
	    - чему посвящена работа в целом,
	    - на чем сконцентрировано исследование,
	    - в чем особенности исследования,
	    - что новое предлагается,
	    - какими примерами проиллюстрирована?
	\fi  
     Работа посвящена предсказанию движений верхних конечностей по данным активности головного мозга, полученным в виде электрокортикограммы (ECoG) или электроэнцефалограммы (EEG). Решается задача прогнозирования многомерных временных рядов.  Исходное пространство избыточно и коррелировано между собой. В работе рассматриваются методы обработки данных, в частности, понижения размерности пространства и отбора признаков, и модели прогнозирования. Предложена система тестирования прогностических моделей для получения предсказаний и анализа качества и ошибки. Приводится сравнение показателей качества рассматриваемых моделей на данных ECoG и EEG.
     
\section{Введение}
	\iffalse
	    - Основное сообщение — чему посвящена работа (одна-две фразы)
        - Обзор литературы — развитие предлагаемой идеи (не более двух абзацев)
        - Современное состояние области (два-четыре абзаца)
        - Что предлагается (два абзаца) 
	\fi
	
	Работа посвящена исследованию методов моделирования нейро-сетевого интерфейса (BCI).
	Входные данные -- сигналы мозга, полученные с помощью электрокортикографии (ECoG) и электроэнцефалографии (EEG). ECoG-сигналы имеют лучшее разрешение и большую амплитуду, однако для их получения требуется непосредственное подсоединение электродов к коре головного мозга. Предлагается декодировать исходные сигналы и спрогнозировать траекторию движения верхних конечностей.
    
    Предсказание намерений может быть полезным при создании нейрокомпьютерного интерфейса.  Подобный инструмент может применяться не только в задаче анализа сигналов мозга, но и во многих других задачах, связанных с прогнозированием многомерных временных рядов.
    
    Исходное пространство имеет избыточно высокую размерность. Для устранения мультиколлинеарности предлагается применить методы понижения размерности и отбора признаков. В скрытом пространстве происходит согласование между моделями.
    
    Признаковое описание многомерного временного ряда существует в пространствах независимых и зависимых переменных. Для учета существующих закономерностей в исходном и выходном пространстве используется скрытое пространство латентных переменных. 
    
    Рассматриваются следующие модели: метод частичных наименьших квадратов (PLS), отбор признаков с помощью квадратичного программирования (QPFS), метод Белсли (Belsley) и вариации этих методов.
    % Нелинейные модели -- нейросети.
    
	Предлагается система тестирования прогностических моделей с оценкой качества и анализом ошибки.

\iffalse	
\begin{thebibliography}{1}
	\bibitem{PCAandfriends}
	\url{https://www.math.uwaterloo.ca/~aghodsib/courses/f06stat890/readings/tutorial_stat890.pdf}
	\bibitem{PLS}
	\url{https://www.utdallas.edu/~herve/Abdi-PLS-pretty.pdf}
	\bibitem{QPFS}
	\url{http://www.jmlr.org/papers/volume11/rodriguez-lujan10a/rodriguez-lujan10a.pdf}
\end{thebibliography}
\fi
\end{document}